the first step we need to take is analyzing the effect of varying a trajectory first. In general, one can expect to vary trajectory followed by the state of a control system in two ways: given a control, varying the initial conditions or, given the initial conditions, varying the control. Nevertheless it is still necessary to develop some tools to \textit{describe} the variation of the trajectory in some ways. 

\subsection{Variations and adjoint response}
\paragraph{Variational and adjoint equations}\mbox{}\\
Given a control system \controlSystem and am admissible control $\mu:I\fd U$ we have

\lista{
	\item the \grass{variational equation}  equation for $\Sigma$ with control $\mu$ is the differential equation
	\begin{multiLineSingleNumber}
		\statotdot=f(\statot,\mut);\\
		\dot{v}(t)=\Dderarg{1}f(\statot,\mut) v(t)\\
		(\statot,\mut)\in(\chi\times\R^n)
		
		\label{variational-equation}
	\end{multiLineSingleNumber}	
	
	
	\item the \grass{variational equation}  equation for $\Sigma$ with control $\mu$ is the differential equation
	\begin{multiLineSingleNumber}
		\statotdot=f(\statot,\mut);\\
		\lambdatdot=-\Dderarg{1}f^T(\statot,\mut) v(t)\\
		(\statot,\lambdat)\in(\chi\times\R^n)
		
		\label{adjoint-equation}
	\end{multiLineSingleNumber}
}
\subparagraph{interpretation} It is straightforward to see that the variational equation describes, through a linearization,  the evolution in time of a small (infinitesimal) variation from the original trajectory $\statot$, which are the solutions of the equation \ref{e1.1}\\
The geometrical interpretation of the adjoint equation is more subtle, but, in a naive way, one could say that, given an optimal trajectory, the adjoint response is a vector orthogonal to the hyerplane given by (directions of the )the possibile (infinitesimal) variations to that trajectory. 

	