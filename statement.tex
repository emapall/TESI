\subsection{Maximum principle for free interval problems}
Let \controlSystem\space be a control system,, $L$ a Lagrangian, \sz\space and \so\space subsets of $\chi$.
A necessary condition for a controlled trajectory $(\xi_*,\mu_*)$ defined on $[t_0,t_1]$  to be optimal, that is, a necessary condition so that $(\xi_*,\mu_*)\in\mathfrak{P}(\Sigma,L,\szm,\som)$, is the existence of an absolutely continuous map $\lambda_*:[t_0,t_1]\fd\R^n$ and of  $\lambda_*^0\in\{-1,0\}$ that have also the following properties
\begin{enumerate}
	\item either $\lambda_*^0=-1$ or $\lambda_*(t_0)\ne0$,
	\item $\lambda_*$ is an adjoint repsonse for $(\Sigma,\lambda_*^0L)\text{ along }(\xi_*,\mu_*)$,
	\item $H_{\Sigma,\lambda_*^0L}(\xi_*(t),\lambda_*(t),\mu_*(t))=H_{\Sigma,\lambda_*^0L}^{max}(\xi_*(t),\lambda_*(t))$ for almost every $t\in[t_0,t_1]$,\\
	If $\mu_*$ is bounded, then 
	\item $\forall t\in\tzto$  $H_{\Sigma,\lambda_*^0L}^{max}(\xi_*(t),\lambda_*(t))=0$.\\
	
	Also, if \so\space and \sz\space are smooth constraint sets, then $\lambda_*$ can be chosen such that 
	\item $\lambda_*(t_0)$ is orthogonal to ker($\Dder\Phi_0(\xi(t_0))$) and $\lambda_*(t_1)$ is orthogonal to ker($\Dder\Phi_1(\xi(t_1))$).
\end{enumerate}
For the fixed interval problem only condition 4 is lost.

\subsection{Maximum principle for fixed interval problems}
Let \controlSystem\space be a control system, $L$ a Lagrangian, \sz\space and \so\space subsets of $\chi;\tzto\subset\R$ an interval.
A necessary condition for a controlled trajectory $(\xi_*,\mu_*)$ defined on $[t_0,t_1]$  to be optimal, that is, a necessary condition so that $(\xi_*,\mu_*)\in\mathfrak{P}(\Sigma,L,\szm,\som,\tzto)$, is the existence of an absolutely continuous map $\lambda_*:[t_0,t_1]\fd\R^n$ and of  $\lambda_*^0\in\{-1,0\}$ that have also the following properties
\begin{enumerate}
	\item either $\lambda_*^0=-1$ or $\lambda_*(t_0)\ne0$,
	\item $\lambda_*$ is an adjoint repsonse for $(\Sigma,\lambda_*^0,L)\text{ along }(\xi_*,\mu_*)$,
	\item $H_{\Sigma,\lambda_*^0L}(\xi_*(t),\lambda_*(t),\mu_*(t))=H_{\Sigma,\lambda_*^0L}^{max}(\xi_*(t),\lambda_*(t))$ for almost every $t\in[t_0,t_1]$.\\
		If $\mu_*$ is bounded, then 
	\item $\forall t\in\tzto$
	$H_{\Sigma,\lambda_*^0L}^{max}(\xi_*(t),\lambda_*(t))$ is constant. \\
	
	Also,if \so\space and \sz\space are smooth constraint sets, then $\lambda_*$ can be chosen such that 
	\item $\lambda_*(t_0)$ is orthogonal to ker($\Dder\Phi_0(\xi(t_0))$) and $\lambda_*(t_1)$ is orthogonal to ker($\Dder\Phi_1(\xi(t_1))$).
\end{enumerate}

%(\xi_*,\mu_*)