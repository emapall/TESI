\subsection{Control System} A \grass{control system} is a triple $\Sigma=(\chi,f,U) $, where 
\begin{enumerate}
	\item $\chi$, representing the states of the system, is an open subset of $\R^n$,
	\item $U$, representing the space of possible (\textit{instantaneous} controls, is an open subset of $\R^m,U\subset\R^m$
	\item $f: \chi \times cl(U) \rightarrow f(x,u)$ is a function which dictates the law with which the system evolves. Moreover, $f$
	\begin{enumerate}
		\item is continuous
		\item the map $x \rightarrow f(x,u)$ is of class $C^1$ for each $u \in cl(U)$
	\end{enumerate}
\end{enumerate}

Since $f$ is function of the current state of the system and of the current control, the evolution of the system does not explicitly depend on time. Of course, the law dictating the evolution of the status of the system is 
\equaz{\statotdot=f(\statot,\mu(t))\label{e1.1}}
where obviously, for each $t,\statot\in\chi$ is the "current" (at time $t$) state of the system, and $\mu(t)\in U$ is the current (at time $t$) control, dictated by the control law $\mu(t)$.


\subsection{Control and Trajectories}
We want some limitations on the function $t\rightarrow\mut$ because, starting from a certain state we want the control to originate through \eqref{e1.1} trajectories that, at least, "make sense". So, given a control system \controlSystemSpaced we define
\begin{itemize}
	\item An \grass{admissible control} as a \textit{measurable map} $\mu:I\rightarrow U$ where $I$ is a (time) interval $I\subset\R$, and such that $t\fd f(x,\mut )$ is locally integrable for each $x\in\chi$.
	\item we denote the set of admissible controls defined on the time interval $I$ by $\admContr{I}$	.
	%https://www.overleaf.com/learn/latex/Mathematical_fonts for reference
	\item A \grass{controlled trajectory} as a pair $(\xi,\mu)$ where, for some time interval $I$
		\lista{
			\item $\mu\in\admContr{I}$ is a function expressing the control through which the system is driven in the time interval, and is an admissible control. This map will be simply called the \grass{control}.
			\item $\xi:I\fd \chi$ is the map linking the times  to their corresponding state, which follows the law \eqref{e1.1}. This map will be called the \grass{trajectory}.
		}
	\item a \grass{controlled arc} is a controlled trajectory defined on a compact time interval.
\end{itemize}
The set of the controlled trajectories for a given control system \controlSystemSpaced is denoted by Ctraj($\Sigma$), the set of the controlled arcs for the control system is denoted by Carc($\Sigma$).


\subsection{Lagrangian, costs and optimal control problem(s)}
Since we want to optimize a cost, we first have to define an objective function which has to be minimized. This will be the integral of another function called Lagrangian. So, given a control system \controlSystem
\begin{itemize}
	\item A \grass{Lagrangian} for $\Sigma\text{ is a function }L:\chi\times cl(U)\fd\R$ such that
		\lista{
			\item L is continuous and
			\item the function $x\fd L(x,u)$ is of class $C^1$ for each $u\in cl(U)$.
		}
	
	\item given a Lagrangian $L$, we say that a controlled trajectory \controlledTrajSpaced with relative time interval \I is \grass{L-acceptable} if the function $t\fd L(\statot,\mut)$ is integrable.
	
	\item given a Lagrangian $L$, the corresponding \grass{objective function} is the map $J_{\Sigma,L}:Ctraj(\Sigma)\fd\overline{\R}$ given by 
	\equaz{J_{\Sigma,L}\controlledTrajMath=\int_I L(\statot,\mut)dt}
	where we set $J_{\Sigma,L}=\infty$ if \controlledTrajSpaced is not L-acceptable.\\
	The set of L-acceptable controlled trajectories (resp .arcs) for the control system is denoted by Ctraj($\Sigma,L$) (resp. Carc($\Sigma,L$)).
\end{itemize}

We should seek to minimize the objective function, with the "parameter" to be tuned being the controlled trajectory \controlledTraj. Usually the problem faced is such that the system will start its evolution in a certain initial state, which lie in a set of possible initial conditions $S_0$, and some end conditions will be given, which means that in the end, the state of the system should lie in another set, $S_1$. Of course $S_0,S_1\subset\chi$. We thus call Carc($\Sigma,L,\szm,\som$) the set of controlled arcs for the control system \controlSystemSpaced with Lagrangian $L$, which have also the following properties:
\lista{
\item every \controlledTrajSpaced in \carcSS\space is defined on a time interval of the form $[t_0,t_1]\subset\R $.
\item if \controlledTraj$\in$\carcSS\space then the  controlled arc is also in Carc($\Sigma,L$), which means that it is an L-acceptable controlled arc.
\item if $\controlledTrajMath \in$\carcSS\space is defined on the time interval $[t_0,t_1]$ then $\chi(t_0)\in\szm$ and $\chi(t_1)\in\som$.}

Now we can precisely define the optimization problem. There are actually two of these problems, depending on the fact that $\tzto$ may or may not be fixed. We are only going to consider the proof for the fixed interval case.\\
\paragraph{Free interval optimal control problem}Let us consider  \lista{
	\item a control system \controlSystem,
	\item a Lagrangian $L$,
	\item $\szm,\som\subset\chi$ sets,
	}
	then a controlled trajectory {$(\xi_*,\mu_*)\in$\carcSS\space is a \grass{solution to the free interval optimal control problem} if $\forall\controlledTrajMath\in$\carcSS,  $\objFunction{(\xi_*,\mu_*)}<\objFunction{\controlledTrajMath}$.\\
	The set of all the possible solutions is denoted by $\mathfrak{P}(\Sigma,L,\szm,\som)$.
\paragraph{Fixed interval optimal control problem}Let us consider  \lista{
	\item a control system \controlSystem,
	\item a Lagrangian $L$,
	\item $\szm,\som\subset\chi$ sets,
	\item a time interval $[t_0,t_1]$
}
	then a controlled trajectory {$(\xi_*,\mu_*)\in$\carcSSfix is a \grass{solution to the fixed interval optimal control problem} if $\forall\controlledTrajMath\in$\carcSSfix, $ \objFunction{(\xi_*,\mu_*)}<\objFunction{\controlledTrajMath}$.\\
	The set of all the possible solutions is denoted by $\mathfrak{P}(\Sigma,L,\szm,\som,[t_0,t_1])$.
\subparagraph{A simple example} The problem in which the cost is the time with which the system is driven from \sz\space to \so is simply a free interval optimal control problem, in which there is a control system with Lagrangian $L(x,u)=1$.


\subsection{Hamiltonians}
The maximum principle is related with the maximization of a Hamiltonian associated with a control system with a certain Lagrangian, so we have the following definitions.\\
Let \controlSystem\space be a control system and $L$ a Lagrangian, then
\lista{
	\item the \grass{Hamiltonian} is the function $H_\Sigma:\chi\times\R^n\times U\fd \R$ given by
		\gatherNo{H_\Sigma(x,p,u)=\langle p,f(x,u)\rangle}
	\item the \grass{extendend Hamiltonian} is the function $H_{\Sigma,L}:\chi\times\R^n\times U\fd \R$ given by
		\begin{equation*}
		H_{\Sigma,L}(x,p,u)=\langle p,f(x,u)\rangle+L(x,u)=H_\Sigma(x,p,u)+L(x,u)
		\end{equation*}
	\item the \grass{maximum Hamiltonian} is the function $H_{\Sigma}^{max}:\chi\times\R^n\fd \overline\R$ given by	
		\gatherNo{H_{\Sigma}^{max}=sup\{H_\Sigma(x,p,u)|u\in U\}}
	\item the \grass{maximum extended Hamiltonian} is the function $H_{\Sigma,L}^{max}:\chi\times\R^n\fd \overline\R$ given by	
		\gatherNo{H_{\Sigma,L}^{max}=sup\{H_{\Sigma,L}(x,p,u)|u\in U\}}
	\item the variable $p$ is sometimes called \grass{costate}.
}


\subsection{Adjoint response}
There are other two quantities, namely the adjoint response and  control variation (with the associated adjoint and variational equation), which are very important for the principle. We will now state the part relative to the adjoint response, because it appears in the enunciate of the principle, while keeping for later the part relative to the control variation. \\
So now let \controlSystem\space be a control system, and \lista{
	\item $\controlledTrajMath\in Ctraj(\Sigma)$ be a controlled trajectory with time interval $I$, then we define the \grass{adjoint response} for $\Sigma$ along \controlledTraj\space as a locally absolutely continuous map $\lambda:I\fd\R^n$ which satisfies the following differential equations:
		\begin{multiLineSingleNumber}
			\statotdot={D}_2H_\Sigma(\statot,\lambdat,\mut)\text{    }(=f(x,u))\\
			\lambdat=-{D}_1H_\Sigma(\statot,\lambdat,\mut)
			\label{adjoint-hamiltonian}
		\end{multiLineSingleNumber}
	Where, given a vector function $a$ of vector variables $x_1,x_2,x_3$, we denote by $\Dderarg{i}a$ the partial derivative with respect to the variable $x_i$. \\
	The above equation can also be expressed in an equivalent form.
	\item if $L$ is a Lagrangian and $\controlledTrajMath\in Ctraj(\Sigma,L)$ is an L-acceptable controlled trajectory with time interval $I$, we then define the \grass{adjoint response} for $(\Sigma,L)$ along \controlledTraj\space as a locally absolutely continous map $\lambda:I\fd\R^n$ which also satisfies the following differential equation(s):
	\begin{multiLineSingleNumber}
		\statotdot={D}_2H_{\Sigma,L}(\statot,\lambdat,\mut)\text{    }(=f(x,u))\\
		\lambdat=-{D}_1H_{\Sigma,L}(\statot,\lambdat,\mut)
	\end{multiLineSingleNumber}  
}


\subsection{Smooth constraint sets}
Part of the maximum principle deals with the case in which \sz\space and \so\space are "smooth", so we might define a \grass{smooth constraint set S} as a subset of the set of the states space$S\subset\chi$ such that there exists a $C^1$ function $\Phi:\chi\fd\R^k$, such that $S=\Phi^{-1}(0)$ and  also $\Dder\Phi(x)$ is surjective for each $x\in S$.

\subsection{Reachable sets}
Let \controlSystem\space be a control system, $x_0\in\chi$ be an initial condition for eq. \eqref{e1.1}, $\tzto\subset\R$ a time interval, $\mu\in\admContr{x_0,t_0,\tzto}$. We then define \lista{
	\item the \grass{reachable set} from $x_0$ at $t_0$ in time $t_1-t_0$ as 
	\[\reachSetMath{t_1}=\{\trajWinCondMath{t_1}|\mu\in\admContr{\tzto} \}  \]
	
	\item the \grass{reachable set } from $x_0$ at $t_0$ as
	\[\mathfrak{R}(x_0,t_0)=\cup_{t_1\in[t_0,\infty]} \reachSetMath{t_1}  \]
	
}
Remark: since $f$ does not depend explicitly on time, then \reachSet{t_1}$=\mathfrak{R}(x_0,0,t_1-t_0)$.